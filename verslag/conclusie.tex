\chapter{Conclusie}
Tijdens dit project werd een schaakcomputer ge\"implementeerd in Prolog. Het algoritme om de beste volgende zet te bepalen is gebaseerd op een variant van minimax-bomen, aangevuld met twee uitbreidingen. Deze uitbreidingen bestaan uit zowel randomisatie, als dynamische diepte. Het ontwikkelingsproces ging relatief vlot, alhoewel debuggen niet altijd even gemakkelijk verliep. Verdere uitbreidingen aan het schaakalgoritme zijn zeker mogelijk, rekeninghoudend met het feit dat professionele schaakcomputers veel dieper gaan dan diepte 3. Deze dieptes kunnen voornamelijk worden bereikt door code optimalisaties alsook betere score functies. Zelf had ik echter geen schaakervaring, dus de huidige scorefunctie leek de meest voordehandliggende. Het werken aan dit project heeft mij veel zinvolle inzichten gegeven in logisch programmeren en in schaken, in de toekomst ben ik zeker van plan Prolog nog te gebruiken voor toepassingen rond artifici\"ele intelligentie.